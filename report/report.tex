%% 
%% Created in 2018 by Martin Slapak
%%
%% Based on file for NRP report LaTeX class by Vit Zyka (2008)

\documentclass[hidelinks, english]{report}

\usepackage[utf8]{inputenc}
\usepackage{url}
\usepackage{mathtools}

\usepackage{graphicx}
\usepackage{blindtext}
\usepackage{dirtree}
\usepackage[export]{adjustbox}
\usepackage{array}
\usepackage{bigstrut}
\usepackage{booktabs}
\usepackage{float}
\usepackage{subfigure}
\usepackage[all]{hypcap}
\usepackage[bottom]{footmisc}
\usepackage{caption}
\usepackage{listings}
\usepackage{cleveref}

\graphicspath{{img/}}

\title{(Real-Time) Gradient-Domain Painting}

\author{Viacheslav Kroilov, Yevhen Kuzmovych}
\affiliation{ČVUT - FIT}
\email{kroilvia@fit.cvut.cz, kuzmoyev@fit.cvut.cz}


\newcommand{\subimage}[3][1]{
\subfigure{
\includegraphics[valign=c, width=#1\textwidth]{#2.#3}
}
}

\newcommand{\smplimage}[3][1]{
\centerline{
\includegraphics[width=#1\textwidth]{#2.#3}
}
}

\newcommand{\image}[4][1]{
\begin{figure}[H]
    \smplimage[#1]{#2}{#3}
    \caption{#4}
    \label{fig:#2}
\end{figure}
}




\begin{document}

\maketitle

%%%%%%%%%%%%%%%%%%%%%%%%%%%%%%%%%%%%%%%%%%%%%%%%%%%%%%%%%%%%%%%%%%%%%%%%%%%%%%%%
\section{Introduction}

This project explores methods for painting in the gradient-domain described in paper by James McCann and
Nancy S. Pollard\cite{gradient}.

In the frameworks of this project, simple GUI application that allows user to paint with gradient-painting brash will
be implemented.

%%%%%%%%%%%%%%%%%%%%%%%%%%%%%%%%%%%%%%%%%%%%%%%%%%%%%%%%%%%%%%%%%%%%%%%%%%%%%%%%
\section{Implementation}

\subsection{Technologies}

Application is implemented in C++ programing language with the usage of the following libraries:

\begin{itemize}
    \item \textbf{Qt framework}. Used for GUI and image processing.
    \item \textbf{amgcl}. Library for solving large sparse linear systems with algebraic multigrid method. It was used
    as a solver on the initial stages of development.
\end{itemize}


%%%%%%%%%%%%%%%%%%%%%%%%%%%%%%%%%%%%%%%%%%%%%%%%%%%%%%%%%%%%%%%%%%%%%%%%%%%%%%%%
\section{Outputs}


%%%%%%%%%%%%%%%%%%%%%%%%%%%%%%%%%%%%%%%%%%%%%%%%%%%%%%%%%%%%%%%%%%%%%%%%%%%%%%%%
\section{Possible improvements}


%%%%%%%%%%%%%%%%%%%%%%%%%%%%%%%%%%%%%%%%%%%%%%%%%%%%%%%%%%%%%%%%%%%%%%%%%%%%%%%%
\section{Conclusion}


%%%%%%%%%%%%%%%%%%%%%%%%%%%%%%%%%%%%%%%%%%%%%%%%%%%%%%%%%%%%%%%%%%%%%%%%%%%%%%%%
\bibliography{reference}

\end{document}
