%% 
%% Created in 2018 by Martin Slapak
%%
%% Based on file for NRP report LaTeX class by Vit Zyka (2008)

\documentclass[hidelinks, english]{report}

\usepackage[utf8]{inputenc}
\usepackage{url}
\usepackage{mathtools}

\usepackage{graphicx}
\usepackage{blindtext}
\usepackage{dirtree}
\usepackage[export]{adjustbox}
\usepackage{array}
\usepackage{bigstrut}
\usepackage{booktabs}
\usepackage{float}
\usepackage{subfigure}
\usepackage[all]{hypcap}
\usepackage[bottom]{footmisc}
\usepackage{listings}
\usepackage{cleveref}
\usepackage{amsmath}
\usepackage{rotating}
\usepackage{multirow}
\usepackage{caption}
\usepackage{arydshln}

\graphicspath{{img/}}

\title{(Nearly almost) Real-Time (but not quite) Gradient-Domain Painting}

\author{Viacheslav Kroilov, Yevhen Kuzmovych}
\affiliation{ČVUT - FIT}
\email{kroilvia@fit.cvut.cz, kuzmoyev@fit.cvut.cz}


\newcommand{\subimage}[3][1]{
\subfigure{
\includegraphics[valign=c, width=#1\textwidth]{#2.#3}
}
}

\newcommand{\smplimage}[3][1]{
\centerline{
\includegraphics[width=#1\textwidth]{#2.#3}
}
}

\newcommand{\image}[4][1]{
\begin{figure}[H]
    \smplimage[#1]{#2}{#3}
    \caption{#4}
    \label{fig:#2}
\end{figure}
}




\begin{document}

    \maketitle

    %%%%%%%%%%%%%%%%%%%%%%%%%%%%%%%%%%%%%%%%%%%%%%%%%%%%%%%%%%%%%%%%%%%%%%%%%%%%%%%%
    \section{Introduction}

    This project explores methods for painting in the~gradient domain described in the~paper by James McCann and
    Nancy S. Pollard\cite{gradient}. When user paints a stroke, gradient is emitted perpendicular to his stroke as
    shown on~\cref{fig:example}.

    In the frameworks of this project, simple GUI application that allows user to draw with gradient-painting brash will
    be implemented.

    \begin{figure}[h]
        \subimage[0.21]{target}{png}
        \subimage[0.21]{result}{png}
        \caption{Gradient-domain painting example.}
        \label{fig:example}
    \end{figure}


    %%%%%%%%%%%%%%%%%%%%%%%%%%%%%%%%%%%%%%%%%%%%%%%%%%%%%%%%%%%%%%%%%%%%%%%%%%%%%%%%
    \section{Methods}
    Gradient painting is effectively the same problem as guided interpolation in image editing. Taking blank image as a
    source ($S$), image with users stroke as a target ($T$), and mask $\Omega$ with the boundary $\delta$ on the whole
    target image except the place of the stroke, finding the resulting image with gradient is equivalent to solving
    a Poisson equation:

    \begin{equation}
        \bigtriangledown^2 u = f
    \end{equation}

    Which in our case means for each pixel $x,y$ in the mask solving:

    \begin{equation}
        \begin{multlined}
            I_{x+1,y} + I_{x-1,y} + I_{x,y+1} + I_{x,y-1} - 4I_{x,y} = \\
            S_{x+1,y} + S_{x-1,y} + S_{x,y+1} + S_{x,y-1} - 4S_{x,y}
        \end{multlined}
        \label{eq:poisson}
    \end{equation}

    where $I$ is the resulting image with boundary conditions on the edges of the image and on the users strokes. Boundary
    conditions means that pixels on $\delta$ are known and $I_{x,y}$ for $ (x,y) \in \delta$ are replaced with
    known $T_{x,y}$.

    Using equation\eqref{eq:poisson} we build sparse matrix $A$ and \textit{right-hand side} vector $B$. Then the solution
    for $AX=B$ is a vector of the pixel values of the resulting image.

    In their paper\cite{gradient}, James McCann and Nancy S. Pollard suggest approximating solution with \textbf{multigrid}
    method, iterative algorithm for solving large sparse linear systems. In the framework of this project, we use
    \textbf{Gauss–Seidel} algorithm.

    %%%%%%%%%%%%%%%%%%%%%%%%%%%%%%%%%%%%%%%%%%%%%%%%%%%%%%%%%%%%%%%%%%%%%%%%%%%%%%%%
    \section{Implementation}

    \subsection{Technologies}

    Application is implemented in C++ programing language with the usage of the following libraries:

    \begin{itemize}
        \item \textbf{Qt framework}. Framework used for GUI and image processing.
        \item \textbf{amgcl}. Library for solving large sparse linear systems with algebraic multigrid method. It was used
        as a solver on the initial stages of the development.
        \item \textbf{CUDA}. Framework for parallelization of the computations on the Nvidia video cards.
    \end{itemize}


    \subsection{Sparse matrix}

    Matrix $A$ has at most 5 non-zero values ($1$s and/or~$-4$) on each row and dimensions of $p \times p$ where $p$ is
    a number of "white" pixels in a mask, which, in our case, are all pixels excluding ones with the stroke. So to minimize
    memory usage and to simplify CUDA parallelization it has structure as shown on figure~\ref{tbl:sparse}.



    \begin{table*}[t]
        \centering
        \begin{tabular}{lclcccccccc}
            & pixels & & \multicolumn{4}{c}{mat}                                                                                                           & \multicolumn{1}{l}{}  & sol & \multicolumn{1}{l}{}  & rhs                        \\ \cline{2-2} \cline{4-7} \cline{9-9} \cline{11-11}
            \multicolumn{1}{l|}{\multirow{4}{*}{unknown}}         & \multicolumn{1}{c|}{$x_0,y_0$}         & \multicolumn{1}{l|}{} & \multicolumn{1}{c|}{$M_{0,0}$} & \multicolumn{1}{c|}{$M_{0,1}$} & \multicolumn{1}{c|}{$M_{0,2}$} & \multicolumn{1}{c|}{$M_{0,3}$} & \multicolumn{1}{c|}{} & \multicolumn{1}{c|}{$I_0$}     & \multicolumn{1}{c|}{} & \multicolumn{1}{c|}{$R_0$} \\ \cline{2-2} \cline{4-7} \cline{9-9} \cline{11-11}
            \multicolumn{1}{l|}{}                                 & \multicolumn{1}{c|}{$x_1,y_1$}         & \multicolumn{1}{l|}{} & \multicolumn{1}{c|}{$M_{1,0}$} & \multicolumn{1}{c|}{$M_{1,1}$} & \multicolumn{1}{c|}{$M_{1,2}$} & \multicolumn{1}{c|}{$M_{1,3}$} & \multicolumn{1}{c|}{} & \multicolumn{1}{c|}{$I_1$}     & \multicolumn{1}{c|}{} & \multicolumn{1}{c|}{$R_1$} \\ \cline{2-2} \cline{4-7} \cline{9-9} \cline{11-11}
            \multicolumn{1}{l|}{}                                 & \multicolumn{1}{c|}{\vdots}            & \multicolumn{1}{l|}{} & \multicolumn{1}{c|}{\vdots}    & \multicolumn{1}{c|}{\vdots}     & \multicolumn{1}{c|}{\vdots}  & \multicolumn{1}{c|}{\vdots}    & \multicolumn{1}{c|}{} & \multicolumn{1}{c|}{\vdots}    & \multicolumn{1}{c|}{} & \multicolumn{1}{c|}{\vdots}\\ \cline{2-2} \cline{4-7} \cline{9-9} \cline{11-11}
            \multicolumn{1}{l|}{}                                 & \multicolumn{1}{c|}{$x_n,y_n$}         & \multicolumn{1}{l|}{} & \multicolumn{1}{c|}{$M_{n,0}$} & \multicolumn{1}{c|}{$M_{n,1}$} & \multicolumn{1}{c|}{$M_{n,2}$} & \multicolumn{1}{c|}{$M_{n,3}$} & \multicolumn{1}{c|}{} & \multicolumn{1}{c|}{$I_n$}     & \multicolumn{1}{c|}{} & \multicolumn{1}{c|}{$R_n$} \\ \cline{2-2} \cline{4-7} \cline{9-9} \cline{11-11}
            \multicolumn{1}{l|}{\multirow{3}{*}{known($\delta$)}} & \multicolumn{1}{c|}{$x_{n+1},y_{n+1}$} & & & & & & \multicolumn{1}{c|}{} & \multicolumn{1}{c|}{$T_{n+1}$} & &                            \\ \cline{2-2} \cline{9-9}
            \multicolumn{1}{l|}{}                                 & \multicolumn{1}{c|}{\vdots}            & & & & & & \multicolumn{1}{c|}{} & \multicolumn{1}{c|}{\vdots}    & &                            \\ \cline{2-2} \cline{9-9}
            \multicolumn{1}{l|}{}                                 & \multicolumn{1}{c|}{$x_p,y_p$}         & & & & & & \multicolumn{1}{c|}{} & \multicolumn{1}{c|}{$T_p$}     & &                            \\ \cline{2-2} \cline{9-9}
            & & & & & & & \multicolumn{1}{c|}{} & \multicolumn{1}{c|}{0}         & &                            \\ \cline{9-9}
        \end{tabular}

        \caption{Sparse matrix structure}
        \label{tbl:sparse}
    \end{table*}

    For $ |\Omega| = p $ and $ |\delta| = p-n $, \textit{pixels} vector stores coordinates of pixels in $\Omega$. $M_{i,j}$
    is an index of $j$s neighbour of pixel $i$. Vector \textit{rhs} contains right-hand side values from
    equation~\eqref{eq:poisson}, but if pixel has one or more neighbours from $\delta$, values of those neighbour-pixels are
    subtracted from the right-hand side. Solutions vector \textit{sol} stores resulting values of pixels from $\Omega$. If
    $j$s neighbour of $i$s pixel is on $\delta$, $M_{i,j}$ is an index of \textit{0} in the end of \textit{sol} (because its
    value is already subtracted from \textit{rhs}).

    So instead of solving matrix $p \times p$, we need to solve matrix $n \times n$ but using all $p$ pixels.

    \subsection{Gauss–Seidel}
    Gauss–Seidel is a simple iterative method for solving linear systems. In our implementation it is defined as
    follows:

    \begin{tabbing}
        1. \=For $it=0$ to number of iterations\\
        \>2. For \=$i=0$ to n $$\\
        \>\>Set
        \begin{math}
            I_i = \dfrac{R_i - \sum_{j=0}^3{I_{M_{i,j}}}}{-4}
        \end{math}
    \end{tabbing}

    The primary problem of this algorithm is that it might not converge to the solution fast enough, which would require
    more iterations and consequently more computation time. The solution is the \textbf{multigrid} method. The idea behind
    multigrid is in precomputing solution on the smaller image and initializing solution vector with precomputed values
    instead of random values (or all 125s as in our implementation). This method allows to perform much more iterations on
    a smaller image in a short period of time and much less iterations on the original image. We scale image recursively
    down to image $5 \times 5$ and perform $\approx 381000$ iterations while on the original image ($300 \times 300$) only
    $\approx 6000$ iterations.


    %%%%%%%%%%%%%%%%%%%%%%%%%%%%%%%%%%%%%%%%%%%%%%%%%%%%%%%%%%%%%%%%%%%%%%%%%%%%%%%%
    \section{Conclusion}

    In frameworks of this project, we've implemented application with graphical interface that allows user to draw in
    the gradient-domain using gradient brash in real-time (not really  $\approx 0.2$ fps).

    Gradient painting is performed using gauss-seidel algorithm with multigrid optimization. Implementation is also
    parallelized with CUDA framework on the GPU .


    %%%%%%%%%%%%%%%%%%%%%%%%%%%%%%%%%%%%%%%%%%%%%%%%%%%%%%%%%%%%%%%%%%%%%%%%%%%%%%%%
    \bibliography{reference}

\end{document}
